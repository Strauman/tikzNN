%!TEX root = main.tex
% Usage:
% \tikzNN[
%   % Prefix of name
%   name={NN},
% 	% How the network is built (number of neurons in index layer)
% 	nn neurons={1,2,3,4},
%   % Default style of layer rectangles
% 	inside layer style={fill=blue!40},
% 	% Styles per layer (the rectangle)
% 	layer styles={{fill=red!40},{fill=green!40}},
% 	% Scale of whole thing
% 	scale={1},
% 	% Distance between neuron and layer rectangle
% 	layer padding={0.05cm},
% 	% How tall the layers are
% 	layer height={3cm},
% 	% Horisontal separation of layers
% 	layer sep={1cm},
% 	% Size of neurons
% 	neuron radius={0.2cm},
% 	% Tikz styles
%   % Style of the connection between neurons
% 	connection style={gray},
% 	% Color of the neurons
% 	neuron color={gray!50},
% 	% Node style for neuron
% 	neuron style={},
%   % Args to send to \begin{scope}
%   tikz args={},
%   % Show the coordinate names of the layers and neurons
%   verbose=false,
% ]


%%%%%%%%%%%%%%%%%%%%%%%%%%%%%%%%%%%%
%%%%%%%%%%%%%%%%%%%%%%%%%%%%%%%%%%%%
%%  NODES defined by this script  %%
%%%%%%%%%%%%%%%%%%%%%%%%%%%%%%%%%%%%
%%%%%%%%%%%%%%%%%%%%%%%%%%%%%%%%%%%%
%% Note that all nodes have coordinates
%% north, south, east and west and
%% combinations of these
%% Rectangle of layer x:
%% NAME-rectangle-x
%% Connection between layer x, neuron y and layer z neuron w:
%% NAME-con-x-y-z-w
%% Rectangle between layer x and y:
%% NAME-between-x-y
%% Bounding box
%% NAME-bb


\makeatletter
\newif\ifNN@hideLayerBox
\newif\iftikzNN@verbose
\providecommand\neuronColor{}
\pgfkeys{
	% /NN/connection style/.code={\pgfkeysalso{/tikz/NN/connection style/.style={#1}}},
	/NN/.cd,
	% See tikzhelpers for info on /.meta append tikz style
	connection style/.meta tikz style={NN/connection style},
	neuron style/.meta append tikz style={NN/neuron style}{fill=\nn@cfg{neuron color}},
	inside layer style/.meta tikz style={NN/inside layer style},
	hide layer boxes/.is if={NN@hideLayerBox},
	user defaults/.settable style,
	user defaults={}
	% this layer style/.meta tikz style={NN/this layer style}
}
\pgfkeys{
	/NN/factory defaults/.style={
			/NN/.cd,
			% TODO: these
			% Name prefix for these tikz figures
			name/.initial={NN},
			input node name/.initial={NN.in},
			output node name/.initial={NN.out},
			% How the network is built
			nn neurons/.initial={4,2,1},
			% Styles per layer (the rectangle)
			layer styles/.initial={,},
			% Scale of whole thing
			scale/.initial={1},
			% Distance between neuron and layer rectangle
			layer padding/.initial={0.05cm},
			% How tall the layers are
			layer height/.initial={3cm},
			% Horisontal separation of layers
			layer sep/.initial={1cm},
			% Size of neurons
			neuron radius/.initial={0.2cm},
			% Tikz styles
			connection style={gray},
			% Color of the neurons
			neuron color/.initial={gray!20},
			% Node style for neuron
			neuron style={},
			% Default style of layer rectangles
			inside layer style={fill=red!70},
			% Args to send to tikz scope
			tikz args/.initial={},
      % Show the coordinate names of the layers and neurons
			verbose/.is if={tikzNN@verbose}
		}
}

% \tnn@nameNode[node settings]{node}
\newcommand\tnn@nameNode[2][]{\node[#1] at (#2){#2};}
% Command for the user to override the defaults
\def\tikzNNDefaults{\pgfqkeys{/NN/user defaults}}
\def\nn@cfg#1{\csname pgfk@/NN/#1\endcsname}
\let\nn@c\nn@cfg
\newif\ifisfirstLayer
\newdimen\firstneuron@x
\newcommand\tikzNN[1][]{%
	% Read keys and set name
	\pgfexpkeys{/NN/.cd,/NN/factory defaults,/NN/user defaults,#1}
	\xdef\@tNN{\nn@c{name}}
	% Do scale and shifting:
	% \edef\scaleArgs{\nn@c{tikz args},scale=\nn@c{scale}, every node/.style={scale=\nn@c{scale}}}
	\edef\scaleArgs{\nn@c{tikz args}, scale=\nn@c{scale},every node/.style={transform shape}}
	\edef\scopeArgs{{scope}[\scaleArgs]}
	\expandafter\begin\scopeArgs

	% \begin{scope}
	% First draw all nodes and such
	\def\layerIdx{0}
	\edef\neuronlist{\nn@c{nn neurons}}
	% To be able to get the last element and such
	\readlist*\neurItemList\neuronlist
	\edef\@styleitms{\nn@c{layer styles}}
	\readlist\layerStylesList\@styleitms
	\providecommand\@fitnodesList{}
	\newcommand\addfitnode[1]{\xdef\@fitnodesList{\unexpanded\expandafter{\@fitnodesList}(##1)}}
	%% Commented below is the old bounding box (didn't work together with scaling?)
	%% \node[inner sep=0pt,fill=none,draw=none,reset transform, fit={(0,0) (\the\dimexpr\neurItemListlen*\nn@c{layer padding}\relax,\nn@c{layer height})}] (NN-BB);

	\foreach\noNeurons in \neuronlist{
		\edef\layerIdxOne{\the\numexpr\layerIdx+1}
		\let\nextLayerIdx\layerIdxOne
		%% All neurons have same X-position in a given layer
		\pgfmathsetlengthmacro\neurX{\nn@c{layer sep}*\layerIdx}
		%% Now figure out how much spacing between neurons
		\pgfmathsetlengthmacro\neuronsep{\nn@c{layer height}/\noNeurons}
		%% Draw rectangle around neurons
		%% Fist set rectangle style for this layer based in the layer styles list
		\pgfkeys{/tikz/NN/this layer style/.style={}}
		\ifnum\layerIdx<\layerStylesListlen\relax
			\edef\thislayerstyleargs{{/tikz/NN/this layer style/.style=\layerStylesList[\layerIdxOne]}}
			\expandafter\pgfkeys\thislayerstyleargs
		\fi
		%% If the `hide layer boxes` option is NOT given
		\ifNN@hideLayerBox\else
			\draw[NN/inside layer style, NN/this layer style]%
			($(\neurX-\nn@c{neuron radius}-\nn@c{layer padding}, 0)$) rectangle node[fitting node](\@tNN-rectangle-\layerIdx){} ++($(2*\nn@c{neuron radius}+2*\nn@c{layer padding},\neuronsep*\noNeurons)$);
			\ifnum\layerIdx > 0\relax
				\edef\prevLayerIdx{\the\numexpr\layerIdx-1}
				% Invisible rectangle in between layers
				\draw[draw=none,fill=none](\@tNN-rectangle-\prevLayerIdx.south east) rectangle node[fitting node](\@tNN-between-\prevLayerIdx-\layerIdx){} (\@tNN-rectangle-\layerIdx.north west);
			\fi
			\addfitnode{\@tNN-rectangle-\layerIdx}
			% Show names of coordinates if verbose
			\iftikzNN@verbose
				\begin{scope}[scale={0.2},every node/.style={scale=0.2}]
					\ifnum\layerIdx > 0\relax%
            \tnn@nameNode[above]{\@tNN-between-\prevLayerIdx-\layerIdx.north};
            \tnn@nameNode[below=2em]{\@tNN-between-\prevLayerIdx-\layerIdx.south}
            \tnn@nameNode[rotate=-90]{\@tNN-between-\prevLayerIdx-\layerIdx.center};
					\fi
					\tnn@nameNode[below]{\@tNN-rectangle-\layerIdx.south};
				\end{scope}
			\fi
		\fi
		% \node[red] at ($(\neurX-\nn@c{neuron radius}, 0)$){X};
		\foreach\neurno in {1,...,\noNeurons}{
				%% Get neuron index
				\edef\neurIdx{\the\numexpr\neurno-1}
				%% Calculate Y-coordinate of neuron
				\pgfmathsetlengthmacro\neurY{\neuronsep*\neurIdx+\neuronsep/2}
				%% Draw neuron and create node
				\node(\@tNN-neur-\layerIdx-\neurIdx) at (\neurX,\neurY){};
				\draw[NN/neuron style,fill=\nn@cfg{neuron color}](\@tNN-neur-\layerIdx-\neurIdx) circle (\nn@c{neuron radius});
				\addfitnode{\@tNN-neur-\layerIdx-\neurIdx}
				\iftikzNN@verbose
					\node[below=\nn@c{neuron radius}/2,scale=0.2] at (\@tNN-neur-\layerIdx-\neurIdx.south){\@tNN-neur-\layerIdx-\neurIdx};
				\fi
			}
		\xdef\layerIdx{\the\numexpr\layerIdx+1}
	}
	\edef\lastLayer{\the\numexpr\layerIdx-1}
	\edef\lastNeuron{\the\numexpr\neurItemList[-1]-1}
	% \draw(NNBB.north west) rectangle (NNBB.south east);
	\edef\numLayers{\the\numexpr\layerIdx+1}
	\edef\numSourceLayers{\the\numexpr\layerIdx-2}
	%% Connect all layers
	%% Draw on background
	\begin{pgfonlayer}{background}
		%% The layersdict will be 1-indexed
		\readlist\layersdict\neuronlist
		\foreach\srcLayer in {0,...,\numSourceLayers}{
				\edef\dstLayer{\the\numexpr\srcLayer+1}
				\edef\numNeuronsInSrcLayer{\layersdict[\the\numexpr\srcLayer+1]}
				\edef\numNeuronsInDstLayer{\layersdict[\the\numexpr\srcLayer+2]}
				\foreach\srcNeuron in {0,...,\the\numexpr\numNeuronsInSrcLayer-1}{
						\foreach\dstNeuron in {0,...,\the\numexpr\numNeuronsInDstLayer-1}{
								\draw[NN/connection style](\@tNN-neur-\srcLayer-\srcNeuron) -- node(\@tNN-con-\srcLayer-\srcNeuron-\dstLayer-\dstNeuron){} (\@tNN-neur-\dstLayer-\dstNeuron);
							}
					}
			}
	\end{pgfonlayer}
	%% Make bounding box coordinate
	%% Calculate width and height of the NN
	\pgfmathsetlengthmacro\Lnnwid{(2*\nn@c{neuron radius}+2*\nn@c{layer padding})+(\neurItemListlen-1)*\nn@c{layer sep}}
	\pgfmathsetlengthmacro\Lnnheight{\nn@c{layer height}}
	\xdef\nnwid{\the\dimexpr\Lnnwid-\nn@c{neuron radius}-\nn@c{layer padding}}
	\xdef\nnheight{\Lnnheight}
	% Obtain the position of the very first node
	\pgfextractx{\firstneuron@x}{\pgfpointanchor{\@tNN-neur-0-0}{center}}%
	\xdef\nnOriginX{\the\dimexpr\firstneuron@x-\nn@c{neuron radius}-\nn@c{layer padding}}
	\xdef\nnOrigin{\nnOriginX,0}
	\coordinate(\@tNN-origin) at (\nnOrigin);
	\coordinate(\@tNN-topRight) at (\nnwid,\nnheight);
	\end{scope}
	\draw[fill=none,draw=none] (\@tNN-origin) rectangle node[fitting node](\@tNN-bb){} (\@tNN-topRight);
}
