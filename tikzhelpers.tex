%!TEX root = example.tex
%%%%%%%%%%%%%%%%%%%%%%%%%%%%%%%%%%%%%%%%%%%%%%%%%%%%%%%%%%%%%%%%%%%%%%%%%%%%
%%%%%%%%%%%%%%%%%%%%%%%%%%%%%%%%%%%%%%%%%%%%%%%%%%%%%%%%%%%%%%%%%%%%%%%%%%%%
%%  This file is part of tikzNN.                                          %%
%%                                                                        %%
%%  tikzNN is free software: you can redistribute it and/or modify        %%
%%  it under the terms of the GNU General Public License as published by  %%
%%  the Free Software Foundation, either version 3 of the License, or     %%
%%  (at your option) any later version.                                   %%
%%                                                                        %%
%%  tikzNN is distributed in the hope that it will be useful,             %%
%%  but WITHOUT ANY WARRANTY; without even the implied warranty of        %%
%%  MERCHANTABILITY or FITNESS FOR A PARTICULAR PURPOSE.  See the         %%
%%  GNU General Public License for more details.                          %%
%%                                                                        %%
%%  You should have received a copy of the GNU General Public License     %%
%%  along with tikzNN.  If not, see <https://www.gnu.org/licenses/>.      %%
%%%%%%%%%%%%%%%%%%%%%%%%%%%%%%%%%%%%%%%%%%%%%%%%%%%%%%%%%%%%%%%%%%%%%%%%%%%%
%%%%%%%%%%%%%%%%%%%%%%%%%%%%%%%%%%%%%%%%%%%%%%%%%%%%%%%%%%%%%%%%%%%%%%%%%%%%

\makeatletter
\usepackage{tikz}
\usepackage{listofitems}
\usepackage{pgfplots}
\usepackage{pgf,tikz}
\usepackage{mathrsfs}
\pgfplotsset{compat=1.16}
\let\dlog\@latex@warning
\newcommand\calcdef[2]{
	\pgfmathparse{#2}\xdef#1{\pgfmathresult}
}
\def\pgfexpkeys#1{
\edef\expanded@keys{{#1}}
\expandafter\pgfkeys\expanded@keys
}
% /NN/neurons/.meta tikz style={tikz/path}
% Now, when /NN/neurons={grey} it also sets /tiks/path/.style={grey}.
\pgfkeys{%
	/handlers/.meta tikz style/.code={%
			\pgfkeys{\pgfkeyscurrentpath/.code={\pgfkeysalso{/tikz/#1/.style={##1}}}}
  },
  % /NN/neurons/.meta append tikz style={tikz/path}{base style}
  % Now, when /NN/neurons={grey} it also sets /tiks/path/.append style={grey}.
  /handlers/.meta append tikz style/.code 2 args={%
		  \pgfkeys{\pgfkeyscurrentpath/.code={\pgfkeys{/tikz/#1/.style={#2}}\pgfkeysalso{/tikz/#1/.append style={##1}}}}
  },
  /handlers/.settable style/.code={\pgfkeys{\pgfkeyscurrentpath/.code=\pgfkeysalso{##1}}}
}
\pgfdeclarelayer{foreground}
\pgfdeclarelayer{background}
\pgfdeclarelayer{wallpaper}
\pgfdeclarelayer{wall}
\pgfsetlayers{wall,wallpaper,background,main,foreground}   %% some additional layers for demo
\usetikzlibrary{fit,calc}
\tikzset{
	fitting node/.style={
			inner sep=0pt,
			fill=none,
			draw=none,
			reset transform,
			fit={(\pgf@pathminx,\pgf@pathminy) (\pgf@pathmaxx,\pgf@pathmaxy)}
		},
	reset transform/.code={\pgftransformreset},
}%
% \usetikzlibrary{arrows}
% \pagestyle{empty}

\newdimen\tmp@tikz@x
\newdimen\tmp@tikz@y
% Extract x and y components from coordinate
% \extractCoords[anchor]{COORDINATE}{outputPrefix}
% Defines \outputPrefix@x, \outputPrefix@y and \outputPrefix@xy -> x,y
\newcommand\extractCoords[3][center]{%
	\pgfextractx{\tmp@tikz@x}{\pgfpointanchor{#2}{#1}}%
	\pgfextracty{\tmp@tikz@y}{\pgfpointanchor{#2}{#1}}%
	\expandafter\xdef\csname #3@xy\endcsname{\the\dimexpr\tmp@tikz@x,\the\dimexpr\tmp@tikz@y}%
	\expandafter\xdef\csname #3@x\endcsname{\the\dimexpr\tmp@tikz@x}%
	\expandafter\xdef\csname #3@y\endcsname{\the\dimexpr\tmp@tikz@y}%
}
\def\excoord(#1){%
\edef\coordname{(#1)}%
\expandafter\coordinate\coordname%
}
\makeatother
